\section{Introduction}

\subsection{Privacy as personal data}
\label{s:privacy}

The purpose of this paper is to discuss different \gls{PPM}s and their possible influence on current business models.
\textit{Privacy} is the central concept in this process, but it is challenging to define.
Cambridge Dictionary \cite{privacy_dict} defines privacy as \textit{someone's right to keep their personal matters and relationships secret}.
However, the definition of privacy heavily depends on the environment, as it is inherently related to a certain culture. 
The way privacy is experienced has roots in philosophy, legality, sociology, politics, and economy \cite{Nissim2018}.

In the past, the concept of privacy was related to one's financial status and wealth. Families living in poverty do not possess the same amount of personal belongings compared to the elite, which meant sharing common items and less privacy.
But privacy has undergone an enormous evolution, ranging from the first privacy-shame dynamic to currently leaning more towards the importance of personal data \cite{Hist2007}. 
Personal data is used as a way to describe privacy nowadays and is in essence information.
This information leads to two crucial and sensitive features: money and power \cite{Hist2007}.

With this in mind, K. D. Martin and P. E. Murphy \cite{Martin2017} propose that it could be considered as a form of currency in exchange for convenience.
Since this is precisely what is required for further discussion, we consider this definition of privacy. 
The usage of personal data is regulated and standardized to some extent due to the introduction of the \gls{GDPR} in 2018. Big Tech companies (Section \ref{s:big tech}) such as Microsoft and Apple, stated in response to this \gls{GDPR} that privacy, defined as the protection of personal data, is a fundamental human right \cite{Schwartz2019}. 

\subsection{The importance of trust}

The sharing of personal data brings numerous challenges. 
On the one hand, it is important to be able to protect oneself without having to be accountable for everything. 
On the other hand, the convenience that comes with sharing your data cannot be underestimated \cite{Hist2007}. 
Moreover, by encroaching on privacy, some benefits can (more easily) be achieved, e.g., video surveillance to provide another type of safety which would not be possible if privacy regulations prevented this \cite{Hist2007}.
Creating a trustworthy environment proves to be essential for an optimal way of processing, analysing and publishing personal data. 
This is especially true for a few specific technological companies that are known as Big Tech. These are the largest and most influential technology companies (Alphabet, Amazon, Apple, Meta and Microsoft in particular), which are further discussed in Section \ref{s:big tech}.
If citizens were to decide not to utilize their services any more due to trust issues, this would result in economic consequences, because their main revenue is based on the personal data of users \cite{Dinev2014}. 

A number of measures have to be met to ensure the user's trust \cite{8906}:

\begin{itemize}
    \item Confidentiality
    \item Authenticity
    \item Integrity
    \item Accountability
    \item Antitrust
\end{itemize}

Confidentiality prevents the disclosure of personal data towards unauthorized parties. 
Users only want to share their personal data with agreed upon parties.
Companies claim that certain values, privacy in this case, are highly important for them. 
Authenticity and integrity, which are closely related, define to which extent these values are fulfilled. 
In addition, accountability assures that companies act responsible. As a result, users are more inclined to put their trust into this company. These measures are exactly why \gls{PPM}s are introduced, which are further discussed in Section \ref{s:types of methods}.

The antitrust law is created as a regulation against the power of singular companies and effectively reducing the power of monopolies \cite{Hart2001}.
This limits the Big Tech companies in their quest for market domination and power by forcing them to split up \cite{Creser2021}. 
Although there is plenty of discussing when it comes to antitrust \cite{Fukuyama2021, Ghosh2021}, this reduces the risks of privacy violations.

Lastly, despite the general belief that people attach great importance to their privacy, some sources seem to disagree.
People might feel confident that they care, but in reality, they are not convinced that this is an issue \cite{Dinev2014}.

\subsection{Overview}
Firstly, the importance of personal data for the Big Tech companies is discussed in Section \ref{s:big tech}. 
Hereafter, an overview of methods used to preserve the privacy of users is described (Section \ref{s:types of methods}).
Section \ref{s:DDBM} touches upon the different business models currently on the market, with a focus on the \gls{DD} methodologies.
Finally, the last section, Section \ref{s:relation PPM}, evaluates the relationship between these methods and business models.
